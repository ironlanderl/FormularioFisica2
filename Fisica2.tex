\documentclass[a4paper,12pt]{article}
\usepackage[margin=0.5cm]{geometry}
\usepackage{multicol}
\usepackage{amsmath}

%disattivo i riferimenti alle formule
\usepackage{mathtools}
\mathtoolsset{showonlyrefs}

\usepackage{amsfonts}
\usepackage{amssymb}
\usepackage{xcolor}
\usepackage{lipsum}
\usepackage[italian]{babel}
\selectlanguage{italian}

\begin{document}
	
	\setlength{\abovedisplayskip}{0pt}
	\setlength{\belowdisplayskip}{0pt}
	
	\title{\Large\bfseries Formulario di Fisica 2\vspace{-1em}}
	\author{Pardini Fabrizio\vspace{-1cm}}
	\date{\today\vspace{-1em}}
	\maketitle
	
	\begin{multicols}{3}
		\setlength{\columnseprule}{0.03pt}
		\def\columnseprulecolor{\color{black}}
		
		\section{Elettrostatica}
		
		
		\begin{equation}
			\text{Legge di coulomb:}F=k*\frac{Q_{1}Q_{2}}{r^{2}}\hat{r}
		\end{equation}
	
		\begin{equation}
			\text{Dove k vale:}k=\frac{1}{4\pi\epsilon_{0}}
		\end{equation}
	
		
		\begin{equation}
			\text{Campo elettrico:}\vec{E}=\frac{\vec{F}}{q}=k\frac{Q}{r^{2}}
		\end{equation}
		
		
		\section{Altra sezione di esempio}
		
		Questa è un'altra sezione di esempio. Inseriamo un'altra formula matematica di fisica, ad esempio la legge di Ohm per la corrente elettrica:
		
		\begin{equation}
			I=\frac{V}{R}
		\end{equation}
		
		Questa formula descrive la relazione tra la corrente elettrica, la differenza di potenziale e la resistenza di un circuito.
		
		
		\section{Lorem}
		\lipsum[1-20]
	\end{multicols}
	
\end{document}
