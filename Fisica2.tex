\documentclass[a4paper,11pt]{article}
\usepackage[margin=0.5cm]{geometry}
\usepackage{multicol}
\usepackage{amsmath}


%disattivo i riferimenti alle formule
\usepackage{mathtools}
\mathtoolsset{showonlyrefs}

\usepackage{amsfonts}
\usepackage{amssymb}
\usepackage{xcolor}
\usepackage{lipsum}
\usepackage[italian]{babel}
\selectlanguage{italian}

\begin{document}
	
	%disattivo lo spazio tra le formule
	\setlength{\abovedisplayskip}{0pt}
	\setlength{\belowdisplayskip}{0pt}
	
	\title{\Large\bfseries Formulario di Fisica 2\vspace{-1em}}
	\author{Pardini Fabrizio\vspace{-1cm}}
	\date{}
	\maketitle
	
	\begin{multicols}{3}
		\setlength{\columnseprule}{0.03pt}
		\def\columnseprulecolor{\color{black}}
		
		\section{Elettrostatica}
		
		\begin{equation}
			Q=\sigma S
		\end{equation}
		
		\begin{equation}
			\text{Legge di coulomb: }F=k\frac{Q_{1}Q_{2}}{r^{2}}\hat{r}
		\end{equation}
	
		\begin{equation}
			\text{Dove k vale }=\frac{1}{4\pi\epsilon_{0}}
		\end{equation}
	
		
		\begin{equation}
			\text{Campo elettrico: }\vec{E}=\frac{\vec{F}}{q}=k\frac{Q}{r^{2}}
		\end{equation}
	
	
		\section{Res, Res e Cond}
		
		\begin{equation}
			R=\frac{V}{i} \left[
			\frac{\text{volt}}{\text{ampere}}=\Omega\text{(ohm)}
			\right]
		\end{equation}
	
		\begin{equation}
			p=\frac{E}{j} \left[
			\frac{V}{m} \cdot \frac{m^{2}}{A} = \Omega \cdot m
			\right]
		\end{equation}
	
		\begin{equation}
			\theta = \frac{1}{p} \left[
			\frac{1}{\Omega \cdot m} = \frac{S\text{(siemens)}}{m}
			\right]
		\end{equation}
	
	
		\section{1^{a} Maxwell}
		
		\begin{equation}
			\text{Gauss: } \phi_{s}\left(\vec{E}\right) = \oint_S\vec{E}\hat{n}dS=\frac{Q_{int}}{\epsilon_{0}}
		\end{equation}
	
		\begin{equation}
			\text{Differenziale: } div\vec{E}_{0} = \frac{\rho}{\epsilon_{0}}
			\text{->} \vec{\nabla}\vec{E}_{0} = \frac{\rho}{\epsilon_{0}}
		\end{equation}
		
		\section{Costanti}
		$\epsilon_{0}=8.854 \cdot 10^{-12}$
		
		
		
		
		\section{Lorem}
		\lipsum[1-10]
	\end{multicols}
	
\end{document}
