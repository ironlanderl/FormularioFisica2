\documentclass[a4paper,11pt]{article}
\usepackage[margin=0.5cm]{geometry}
\usepackage{multicol}
\usepackage{amsmath}
\usepackage{forloop}

%disattivo i riferimenti alle formule
\usepackage{mathtools}
\mathtoolsset{showonlyrefs}

\usepackage{amsfonts}
\usepackage{amssymb}
\usepackage{xcolor}
\usepackage{lipsum}
\usepackage[italian]{babel}
\selectlanguage{italian}

\begin{document}
	
	%disattivo lo spazio tra le formule
	\setlength{\abovedisplayskip}{0pt}
	\setlength{\belowdisplayskip}{0pt}
	
	\title{\Large\bfseries Formulario di Fisica 2\vspace{-1em}}
	\author{Pardini Fabrizio\vspace{-1cm}}
	\date{}
	\maketitle
	
	\begin{multicols}{3}
		\setlength{\columnseprule}{0.03pt}
		\def\columnseprulecolor{\color{black}}
		
		\section{Elettrostatica}
		
		\begin{equation}
			Q=\sigma S
		\end{equation}
		
		\begin{equation}
			\text{Legge di coulomb: }F=k\frac{Q_{1}Q_{2}}{r^{2}}\hat{r}
		\end{equation}
		
		\begin{equation}
			\text{Dove k vale }=\frac{1}{4\pi\epsilon_{0}}
		\end{equation}
		
		
		\begin{equation}
			\text{Campo elettrico: }\vec{E}=\frac{\vec{F}}{q}=k\frac{Q}{r^{2}}
		\end{equation}
		
		
		\section{Res, Res e Cond}
		
		\begin{equation}
			R=\frac{V}{i} \left[
			\frac{\text{volt}}{\text{ampere}}=\Omega\text{(ohm)}
			\right]
		\end{equation}
		
		\begin{equation}
			p=\frac{E}{j} \left[
			\frac{V}{m} \cdot \frac{m^{2}}{A} = \Omega \cdot m
			\right]
		\end{equation}
		
		\begin{equation}
			\theta = \frac{1}{p} \left[
			\frac{1}{\Omega \cdot m} = \frac{S\text{(siemens)}}{m}
			\right]
		\end{equation}
		
		
		\section{1° Maxwell}
		
		\begin{equation}
			\text{Gauss: } \phi_{s}\left(\vec{E}\right) = \oint_S\vec{E}\hat{n}dS=\frac{Q_{int}}{\epsilon_{0}}
		\end{equation}
		
		\begin{align*}
			\text{Differenziale: } div\vec{E}_{0} &= \frac{\rho}{\epsilon_{0}} \\
			\Rightarrow \vec{\nabla}\vec{E}_{0} &= \frac{\rho}{\epsilon_{0}}
		\end{align*}
		
		
		
		
		
		\section{Potenziale elettrostatico}
		
		\begin{equation}
			\text{Nel vuoto: } \vec{E}_{0}\left(\vec{r}\right) = k\frac{Q}{r^{2}}\hat{r}
		\end{equation}
		
		\begin{align*}
			\int_{A}^{B} \vec{E}_{0} \cdot d\vec{l} &= V_0(A) - V_0(B) = -\Delta V \\
			&\Rightarrow \vec{E}_{0} \cdot d\vec{l} = -dV_{0} \\
			&\Rightarrow \vec{E}_0 = -\vec{\nabla} V_0 \equiv -\vec{\mathrm{grad}} V_0
		\end{align*}
		
		\begin{equation}
			\text{Potenziale generato: } V_0(r) = k\frac{Q}{r}
		\end{equation}
		
		
		\section{Dipolo elettrico}
		
		\begin{align*}
			V(P) &= \frac{q}{4\pi\epsilon_0}\left(\frac{1}{r_2}-\frac{1}{r_1}\right) \\
			&\Rightarrow kq\frac{r_1^2-r_2^2}{r_1r_2\left(r_1+r_2\right)} \\
			&\Rightarrow \frac{1}{4\pi\epsilon_0}\left(\vec{p}\cdot\vec{r}\right)\frac{1}{r^3}
		\end{align*}
		
		\begin{align*}
			\vec{E} &= \frac{1}{4\pi\epsilon_0} \left(\frac{3(\vec{p}\cdot\vec{r})\vec{r}}{r^5} - \frac{\vec{p}}{r^3}\right) \Rightarrow \\
			E_r &= \frac{2kp\cos\theta}{r^3} \\
			E_\theta &= \frac{kp\sin\theta}{r^3}			
		\end{align*}
		
		\section{Conservatività del campo elettrico}
		
		\begin{equation}
			\oint\vec{E}_0d\vec{l}=0
		\end{equation}
	
		\begin{equation}
			\text{rot}\vec{E}_0 \equiv \nabla \wedge \vec{E}_0 = 0
		\end{equation}
	
	
		\section{Condensatori}
		\begin{equation}
			\text{Capacità: } C=\frac{Q}{V}
		\end{equation}
		
		\begin{equation}
			\text{Parallelo: } C_{tot} = \sum_{i=1}^{n}C_{i}
		\end{equation}
	
		\begin{equation}
			\text{Serie: } \frac{1}{C_{tot}} = \sum_{i=1}^{n}\frac{1}{C_{i}}
		\end{equation}
	
		\begin{equation}
			U = L = \frac{1}{2} \int_{S}\sigma VdS
		\end{equation}
	
		\section{Dielettrici}
		\begin{equation}
			C = \epsilon_rC_0 = \epsilon\frac{S}{d}
		\end{equation}
	
		\begin{equation}
			\text{Polarizzazione: } <\vec{p}> = \alpha_0\vec{E} = \frac{p_0^2\vec{E}}{3k_BT}
		\end{equation}
	
		\begin{equation}
			V = \frac{1}{4 \pi \epsilon_0} \frac{\vec{P}\vec{r}}{r^3}
		\end{equation}
	
		\section{Polarizzazione Dielettrica}
		\begin{equation}
			\vec{P_e} = \frac{\sum_{i=1}\vec{P_i}}{\delta V} \left[\frac{C}{m^2}\right]
		\end{equation}
	
		\begin{align*}
			V &= \vec{d} \vec{\delta S} \\
			N &= nV = n\vec{d} \vec{\delta S} \\
			\sigma_p &= \frac{Q_{Tot}}{S} = \frac{Nq}{S} = ndqcos\theta = \vec{P_e} \cdot \hat{n}
		\end{align*}
	
		\section{CAMPO MAGNETICO COSTANTE NEL VUOTO}
	
	
	
	
	
	
	
	
		
		
		\section{Costanti}
		$\epsilon_{0}=8.854 \cdot 10^{-12}$
		
		
		
		\section{Altro}
		
		\begin{equation}
			\vec{\nabla} \equiv
			\frac{\partial}{\partial x}\hat{i} +
			\frac{\partial}{\partial y}\hat{j} +
			\frac{\partial}{\partial z}\hat{k}
		\end{equation}
		
		\begin{equation}
			\text{Momento di dipolo elettrico: }\vec{p} \equiv q \vec{d}
		\end{equation}
		
		\begin{equation}
			\text{Densità: } \sigma = \frac{q}{S}
		\end{equation}
		
		\begin{equation}
			\text{Due piani paralleli: }\vec{E}_0 = \frac{\sigma}{\epsilon_0} \hat{i}
		\end{equation}
		
		
		
		
		
		\section{Lorem}
		\lipsum[1-10]
	\end{multicols}
	
\end{document}
